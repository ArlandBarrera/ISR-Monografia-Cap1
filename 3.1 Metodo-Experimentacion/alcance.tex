La investigación es de tipo experimental con el diseño de un prototipo funcional, que emplea tecnologías de reconocimiento de imágenes y sensores ultrasónicos para clasificar de manera automática los residuos. Esto permite validar la hipótesis mediante pruebas en condiciones reales, evaluando la efectividad del sistema en clasificar diferentes tipos de materiales y contribuyendo a la reducción de residuos no clasificados correctamente.

La elección de un enfoque experimental obedece a la necesidad de observar y medir la precisión del sistema en un entorno controlado antes de una posible implementación a gran escala. Este tipo de investigación también permite hacer ajustes y mejoras al prototipo en función de los resultados obtenidos durante las pruebas.


El alcance del proyecto es de prueba de concepto, es decir, se busca construir un prototipo funcional que demuestre la viabilidad y eficacia de una solución automatizada para la clasificación de residuos. Al limitar el alcance a una prueba de concepto, se facilita el desarrollo inicial, y se enfoca en comprobar si la tecnología y el diseño propuestos cumplen con los objetivos de clasificación eficiente y mejora en la gestión de residuos.

Este prototipo servirá como base para evaluaciones futuras, pudiendo ser optimizado para aplicaciones más amplias en el campus o en otros contextos similares. Al finalizar el proyecto, se espera contar con datos que validen la hipótesis y ofrezcan una visión clara de los beneficios potenciales de la tecnología aplicada.