\begin{table}[H]
    \centering
    \begin{tabular}{|l|c|c|c|}
        \hline
        \textbf{Actividad} & \textbf{Duración Est.} & \textbf{Fecha Inicio} & \textbf{Fecha Fin} \\
        \hline
        Investigación inicial y planificación & 2 semanas & 11/9/2024 & 24/9/2024 \\
        \hline
        Diseño conceptual del prototipo & 2 semanas & 25/9/2024 & 8/10/2024 \\
        \hline
        Adquisición de componentes & 1 semana & 9/10/2024 & 15/10/2024 \\
        \hline
        Desarrollo y programación del sistema & 1 semana & 16/10/2024 & 22/10/2024 \\ % Duración ajustada a 1 semana
        \hline
        Ensamblaje del prototipo & 2 semanas & 23/10/2024 & 5/11/2024 \\ % Sin cambios
        \hline
        Pruebas y ajustes del prototipo & 1 semana & 6/11/2024 & 12/11/2024 \\ % Duración ajustada a 1 semana
        \hline
        Análisis de resultados & 2 dias & 13/11/2024 & 14/11/2024 \\ % Ajustado para seguir la secuencia
        \hline
    \end{tabular}
    \caption{Cronograma de Actividades}
    \label{tab:cronograma_actividades}
\end{table}


\begin{itemize}
    \item \textbf{Investigación inicial y planificación}
    \begin{itemize}
        \item \textbf{Duración}: 2 semanas (11/9/2024 - 24/9/2024)
        \item \textbf{Descripción}: Esta fase la consideramos primordial porque nos apoyamos de hombros de gigantes e implica la recopilación de información relevante sobre el proyecto y la identificación de los objetivos y metas. Se realiza una revisión de la literatura existente y se elabora un plan de acción que guiará el desarrollo del prototipo. Con esto podemos avanzar a las etapas de diseño y ejecución.
    \end{itemize}
    
    \item \textbf{Diseño conceptual del prototipo}
    \begin{itemize}
        \item \textbf{Duración}: 2 semanas (25/9/2024 - 8/10/2024)
        \item \textbf{Descripción}: En esta etapa, se desarrollan diferentes versiones del diseño inicial del prototipo. Esto es como van interactuando los componentes en el prototipo y cual es la salida o solucion brindada al usuario. Se busca definir cómo interactuarán los distintos elementos del sistema, asegurando que cumplan con los requisitos establecidos durante la fase de investigación.
    \end{itemize}

    \item \textbf{Adquisición de componentes}
    \begin{itemize}
        \item \textbf{Duración}: 1 semana (9/10/2024 - 15/10/2024)
        \item \textbf{Descripción}: Durante esta fase, se compran y obtienen todos los componentes necesarios para la construcción del prototipo. Esto incluye el hardware que se veerá en la otra sección(sensores, microcontroladores, etc.) como el software necesario para el desarrollo. Es crucial asegurarse de que todos los elementos estén disponibles para la fase de desarrollo.
    \end{itemize}

    \item \textbf{Desarrollo y programación del sistema}
    \begin{itemize}
        \item \textbf{Duración}: 1 semana (16/10/2024 - 22/10/2024)
        \item \textbf{Descripción}: Esta etapa implica la codificación del software y la integración de los diferentes componentes del sistema. Se desarrollan las funciones necesarias y se realizan pruebas unitarias para asegurar que cada parte del sistema funcione correctamente. Es una fase crítica, ya que una buena programación facilitará la posterior integración y pruebas del prototipo.
    \end{itemize}
    \newpage
    \item \textbf{Ensamblaje del prototipo}
    %\vspace{-15mm}
    \begin{itemize}
        \item \textbf{Duración}: 2 semanas (23/10/2024 - 5/11/2024)
       % \vspace{-7mm}
        \item \textbf{Descripción}: En esta fase, se ensamblan físicamente todos los componentes del prototipo según el diseño conceptual. Se presta atención a como van a estar los elementos, la conexión de cables y la implementación del sistema en un formato compacto. Este paso es esencial para garantizar que el sistema esté listo para las pruebas.
    \end{itemize}
   %\vspace{-25mm}
    \item \textbf{Pruebas y ajustes del prototipo}
    %\vspace{-3mm}
    \begin{itemize}
        \vspace{-1.1cm}
        \item \textbf{Duración}: 1 semana (6/11/2024 - 12/11/2024)
        %\vspace{-1.3mm}
        \item \textbf{Descripción}: Durante esta etapa, se realizan muchas pruebas para identificar fallos o áreas que vamos a tener que mejorar en el prototipo. Se ajustan parámetros, se corrigen errores y se optimiza el rendimiento del sistema. Las pruebas son vitales para asegurarnos que el prototipo cumpla con lo que queremos.
    \end{itemize}
   % \vspace{-1mm}
    \item \textbf{Análisis de resultados}
    \begin{itemize}
        \item \textbf{Duración}: 2 días (13/11/2024 - 14/11/2024)
        \item \textbf{Descripción}: En esta última fase, se evalúan los resultados obtenidos de las pruebas. Se analizan los datos recopilados y se determinan las conclusiones sobre el desempeño del prototipo. Este análisis es crucial para decidir si se requiere una revisión adicional del diseño o si el prototipo está listo para su presentación o implementación.
    \end{itemize}
\end{itemize}
