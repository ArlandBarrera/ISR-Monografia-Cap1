\section{Delimitaciones}
El objetivo del proyecto es desarrollar un sistema de clasificación de residuos sólidos reciclables, enfocado específicamente en tres tipos de materiales: papel, cartón y plástico. Si bien esta selección permite abordar una parte importante de los residuos generados, limita la aplicabilidad del sistema a otros tipos de desechos, como metales, vidrios u orgánicos, restringiendo su capacidad para ser utilizado en escenarios de mayor diversidad de residuos.

La implementación tecnológica del sistema se sustenta en la integración de componentes de hardware, como ESP32-CAM y Arduino, lo que ofrece ventajas en términos de programación y control. Sin embargo, esta elección puede plantear desafíos en cuanto al costo, especialmente si se busca replicar el proyecto a gran escala. Además, el éxito del sistema está condicionado por las características del entorno donde se instale, lo que puede limitar su utilización en áreas con restricciones presupuestarias o de infraestructura tecnológica. 

Se plantea un enfoque particular en el Centro Regional de Veraguas de la Universidad Tecnológica de Panamá (UTP), lo que implica que su aplicación se encuentra restringida a un entorno académico específico. Si bien este espacio ofrece condiciones controladas para la prueba y validación del sistema, la escala de implementación es limitada, lo que sugiere la necesidad de futuras adaptaciones para extender su uso a otras instituciones o contextos urbanos más amplios.