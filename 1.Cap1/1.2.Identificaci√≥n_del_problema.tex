\section{Identificación del problema}
De acuerdo con el Programa de Naciones Unidas para el Desarrollo (PNUD) en su documento “Un país pequeño y un reto enorme: La gestión integral de residuos sólidos en Panamá”, en el país con 4,500,000 millones de habitantes cada persona genera 1.2 kilogramos de residuos sólidos por persona al día de los cuales solo el 65 por ciento termina en sitios de disposición final. El resto va a vertederos ilegales, ríos, quebradas, el mar y otros sitios no destinados a la disposición de desechos.~\cite{pnud}

Uno de los principales problemas radica en que los materiales reciclables a menudo no se separan adecuadamente desde el origen, dificultando su reciclaje y aumentando la contaminación. Los sistemas tradicionales no detectan cuando los contenedores están llenos, lo que puede causar desbordamientos y acumulación de desechos.

Esta propuesta del prototipo consiste en un sistema autónomo basado en Arduino y ESP32 para detectar y clasificar materiales de desecho en un cubo de basura inteligente. El cubo está equipado con un módulo de cámara ESP32-CAM para identificar el tipo de material, y un sensor de que monitorea el nivel de llenado.

El sistema procesará los datos en tiempo real mediante la placa Arduino, determinando el tipo de basura a depositar y el estado de llenado del cubo. Cuando se detecte que el cubo está lleno o un material específico, se enviarán notificaciones a través de la ESP32