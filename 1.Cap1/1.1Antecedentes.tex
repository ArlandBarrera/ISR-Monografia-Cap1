\section{Antecedentes}
Revisando en la literatura trabajos similares que se han realizado antes:

\subsection{Sistema de clasificación inteligente de residuos}
"INTELLIGENT WASTE SORTING SYSTEM: LEVERAGING ARDUINO FOR AUTOMATED TRASH IDENTIFICATION AND CATEGORIZATION". El artículo presenta un sistema inteligente para la clasificación automática de residuos, utilizando modelos de detección de objetos basados en YOLOv5x y tecnología Arduino. La propuesta surge debido al aumento de residuos como consecuencia de la urbanización, lo que genera un desafío para el medio ambiente y la gestión de desechos. Se busca optimizar el proceso de detección y clasificación de residuos con el fin de mejorar su reutilización y reciclaje.

Para lograrlo, se entrena un modelo YOLOv5x utilizando un conjunto de datos personalizados que contiene imágenes de siete tipos de residuos, como plástico, vidrio, papel y metales. El sistema utiliza un microcontrolador Arduino para interactuar con el entorno físico, permitiendo que el sistema realice acciones como la clasificación automática de residuos. Los resultados muestran que la combinación de YOLO y Arduino ofrece una solución eficiente para la clasificación de basura, con el potencial de ser implementada en diversos contextos, desde espacios públicos hasta plantas de tratamiento de residuos.~\cite{mohd}

\subsection{Sistema de gestión de residuos inteligente basado en IoT}
"An Internet of Things based Smart Waste System". Presenta un sistema de gestión de residuos basado en la tecnología de Internet de las Cosas (IoT) utilizando el microcontrolador ESP-32 Wi-Fi. El objetivo del sistema es evitar la acumulación de basura en las calles, reducir la carga laboral de los recolectores y automatizar el proceso de vaciado de los contenedores. Los contenedores inteligentes incorporan sensores ultrasónicos para medir el nivel de residuos, sensores DHT-22 para monitorear la humedad y temperatura, y motores servo para abrir y vaciar los contenedores automáticamente. Los datos se muestran en una pantalla LCD y se transmiten a una aplicación móvil a través de IoT, permitiendo la supervisión remota del sistema. El sistema es eficiente y económico, lo que lo convierte en una solución viable para mantener las ciudades limpias y libres de contaminación.~\cite{jasim}

\subsection{Sistema de contenedores de basura basado en IoT que utiliza un microcontrolador NodeMCUu ESP32}
"IOT-BASED GARBAGE CONTAINER SYSTEM USING NODEMCU ESP32 MICROCONTROLLER". Presenta un sistema de contenedores de basura inteligentes basado en la tecnología de Internet de las Cosas (IoT) y el microcontrolador NodeMCU ESP32. Este sistema automatiza la apertura y cierre de la tapa de los contenedores de basura, y permite detectar cuando están llenos. Los usuarios pueden monitorear en tiempo real el nivel de llenado a través de una página web y recibir notificaciones mediante Telegram. Los resultados de las pruebas indican que el sistema mejora la comodidad y cumplimiento de los usuarios al disponer de basura, y facilita la gestión de residuos al informar a los trabajadores cuándo vaciar los contenedores. El uso de IoT para este fin optimiza el manejo de los residuos, evita desbordes y mantiene el entorno limpio.~\cite{anggrawan}