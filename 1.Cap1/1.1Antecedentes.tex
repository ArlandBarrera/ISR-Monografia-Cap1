\section{Antecedentes}
Revisando en la literatura trabajos que se ha realizado antes:

\subsection{Sistema de clasificación inteligente de residuos}
"INTELLIGENT WASTE SORTING SYSTEM: LEVERAGING ARDUINO FOR AUTOMATED TRASH IDENTIFICATION AND CATEGORIZATION". El artículo presenta un sistema inteligente para la clasificación automática de residuos, utilizando modelos de detección de objetos basados en YOLOv5x y tecnología Arduino. La propuesta surge debido al aumento de residuos como consecuencia de la urbanización, lo que genera un desafío para el medio ambiente y la gestión de desechos. Se busca optimizar el proceso de detección y clasificación de residuos con el fin de mejorar su reutilización y reciclaje.

Para lograrlo, se entrena un modelo YOLOv5x utilizando un conjunto de datos personalizados que contiene imágenes de siete tipos de residuos, como plástico, vidrio, papel y metales. El sistema utiliza un microcontrolador Arduino para interactuar con el entorno físico, permitiendo que el sistema realice acciones como la clasificación automática de residuos. Los resultados muestran que la combinación de YOLO y Arduino ofrece una solución eficiente para la clasificación de basura, con el potencial de ser implementada en diversos contextos, desde espacios públicos hasta plantas de tratamiento de residuos.~\cite{mohd}